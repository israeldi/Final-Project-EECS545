%% LyX 2.2.3 created this file.  For more info, see http://www.lyx.org/.
%% Do not edit unless you really know what you are doing.
\documentclass[english]{article}
\usepackage[T1]{fontenc}
\usepackage[latin9]{luainputenc}
\usepackage{babel}
\begin{document}

\section{(Title and Abstract) Make sure that author list is anonymized. }

Introduction or Problem statement: 

a. What is the topic or goal of your project? Why do you think it
is interesting or worth pursuing? Why do you think it is not trivial? 

b. What would be the novelty or contribution of your project? 

\section{Method: Proposed approach or Methods to be developed }

a. {[}For the Implementation track{]} Explain the method (the high-level
idea and optionally the technical part) on your own words, and background
as much as needed (assuming the reader would have understood the topics
covered in this class). A blind copy-and-paste is not allowed. Discuss
why and how good this method is. 

b. {[}For the Application track{]} Which machine learning method is
going to be applied? 

c. {[}For the Open-Ended track{]} Explain the method or approach you
are developing. It is fine to not have a complete description. 

\section{Related work}

a. Discuss related works (at least 5 references). A comprehensive
survey of related works (either methods or practical applications)
would be useful. It is important to elaborate on how they would be
different from or related to your project. 

b. For example, projects in the Application track might discuss relevant
works on the dataset, similar analysis. Projects in the Open-ended
track should discuss relevant prior approaches and discuss how they
are different from and related with your method. 

\section{(Plan of) Experiments }

a. How will you implement your work? List some software to be used
or developed. What software or off-the-shelf tools can be used? What
are you going to create on your own? 

b. Dataset to be used (real-world data or simulated). You can discuss
the nature of the dataset. How can it be obtained publicly, or how
you collect/generate the data? What are the statistics of the dataset?
Why is it challenging, or why would it be useful/interesting to study? 

c. Evaluation. How they can be evaluated? What experimental settings,
dataset, evaluation metrics will you use? 

\section*{Plan of Project }

a. Anticipated division of work over team. 

b. Why do you think it is accomplishable and feasible in a given timeframe?
Discuss briefly. What challenges and difficulties would you expect? 
\end{document}
